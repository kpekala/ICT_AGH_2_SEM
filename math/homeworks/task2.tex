\documentclass[12 pt]{article}        	%sets the font to 12 pt and says this is an article (as opposed to book or other documents)
\usepackage{amsfonts, amssymb}					% packages to get the fonts, symbols used in most math
\usepackage{polski}
%\usepackage{setspace}               		% Together with \doublespacing below allows for doublespacing of the document

\oddsidemargin=-0.5cm                 	% These three commands create the margins required for class
\setlength{\textwidth}{6.5in}         	%
\addtolength{\voffset}{-20pt}        		%
\addtolength{\headsep}{25pt}           	%



\pagestyle{myheadings}                           	
\markright{Konrad Pękala\hfill \today \hfill} 


\begin{document}									
Wiemy, że \(x,y,z \in \{0, 1\}\) i, że \(xy = z\).
Wypiszmy wszystkie możliwe wartości w tabelce: 
\begin{center}
\begin{tabular}{ |c|c|c| } 
\hline
 x & y & z \\
 \hline
 0 & 0 & 0 \\
 \hline
 0 & 1 & 0 \\
 \hline
 1 & 0 & 0 \\
 \hline
 1 & 1 & 1 \\
 \hline
\end{tabular}
\end{center}
Zajmijmy się ograniczeniami: \\
\(z \leq x\) Jeśli \(x=0\) to \(z=0\) \\
\(z \leq y\) Jeśli \(y=0\) to \(z=0\) \\
\(z \geq x + y - 1\) Jeśli \(x = 1\) lub \(y = 1\) to \(z = 1\)
\end{document}